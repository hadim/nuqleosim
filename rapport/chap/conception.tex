\chapter{Conception}

Dans ce chapitre, le fonctionnement de l'application sera décrit de
manière détaillée, tout d'abord pour la partie base de données, puis
pour la modélisation et enfin pour l'interface.

\section{Base de données}

La base de données doit contenir deux types très différents de
données. Des expériences, principalement des photos de microscopie,
ainsi que des molécules. La base de données a donc été conçue de manière
à séparer ces deux types de données.

L'élément \texttt{Molecule} contient les données relatives aux
molécules. Afin d'assurer une interopérabilité des informations, les
identifiants de trois grandes bases de données biologiques ont été
inclus (Protein Data Bank (PDB), EMBL et UniprotKB). D'autres liens vers des bases de données peuvent être ajoutés dans le cas d'une poursuite du
développement de \NQ.

L'élément \texttt{Experiment} contient, lui, les données relatives à des
expériences. Les données sont principalement une liste d'images ainsi
que des données courantes telles que l'auteur et la date. 

Un champ supplémentaire, \texttt{comments} est présent pour chacune des deux classes d'éléments et permet à l'utilisateur de \NQ de laisser des informations complémentaires de différents types.

Afin de répondre à toutes ces caractéristiques, la base de données de
\NQ est organisée selon le schéma qui suit :

\image{diagDB}{Schéma de la base de données.}{0.5}

\section{Modélisation du nucléole}

\subsection{Modélisation des composants du nucléole}

Le nucléole est une structure du noyau des cellules eucaryotes, mais
ce n'est pas un compartiment. En effet, aucune paroi ne le délimite,
seule sa composition permet de l'identifier. Comme il a été vu
précédemment, il est le centre de la production intense des
ribosomes. Sa densité est donc supérieure à celle du reste du noyau et
son apparence en microscopie est, par conséquent, elle aussi
différente (\cite{Raska2006} et \cite{Thiry2005}).

La localisation dans le noyau du nucléole n'ayant aucun impact
important pour la réalisation de ce projet, la modélisation du
nucléole commence donc par la modélisation de ce composant
lui-même. En outre, il a été observé que le nucléole lui-même comptait
plusieurs composants bien distincts. Il comprend le DFC (\textit{Dense
  Fibrillar Component}), le FC (\textit{Fibrillar Component}) et GC
(\textit{Granular Center}). Cependant ce sont des composants et non
des compartiments, c'est donc leur composition qui les caractérise et
non la présence d'une membrane délimitant les divers compartiments. Le
nucléole modélisé doit prendre en compte cette spécificité. Il doit
donc comprendre trois zones distinctes représentant respectivement les
trois composants.

Comme cela vient d'être vu, chaque composant du nucléole se distingue
de par sa composition en protéines et en ARN. Cette caractéristique
doit donc, elle aussi, être prise en compte. C'est-à-dire qu'il doit
être possible de préciser, pour chaque molécule ayant un rôle dans la
simulation, les composants dans lesquels elle se déplace. Enfin, c'est
au coeur du nucléole que se déroule la transcription des ARN
ribosomiaux. Il y a donc synthèse puis maturation de ces ARN dans les
différents composants du nucléole. Ceci constitue le dernier point à
modéliser pour mettre en place une simulation. En effet, les ARN sont
transcrits au niveau du FC. Puis, ils subissent plusieurs maturations
en traversant le DFC puis le GC, pour sortir du nucléole et du noyau et
être assemblés en ribosomes. La modélisation du nucléole doit donc
également prendre en compte cette autre caractéristique.

En résumé, la modélisation du nucléole doit prendre en compte les
caractéristiques de l'environnement \og{}nucléole\fg{} :

\begin{itemize}
	\item zone délimitant la composante FC
	\item zone délimitant la composante DFC
	\item zone délimitant la composante GC
	\item position des molécules dans l'environnement : deux molécules
      ne peuvent pas se superposer.
\end{itemize}

Cette modélisation doit également prendre en compte les molécules
intervenant dans la simulation :

\begin{itemize}
\item protéine: sa position, sa vitesse de déplacement, les
  composantes dans lesquelles elle prend part.
\item ARN : sa position, sa vitesse de déplacement, les composantes
  dans lesquelles elle prend part et son état de maturation, sa
  vitesse de maturation.
\item \og{}transcriptor\fg{} : sa position dans le FC, sa vitesse de
  déplacement, sa vitesse de \og{}transcription\fg{}.
\end{itemize}

\subsection{Approche par système multi-agents}

Un système multi-agents (SMA) est un système composé d'un ensemble
d'agents, situés dans un certain environnement et interagissant selon
certaines relations. Un agent est une entité caractérisée par le fait
qu'elle est, au moins partiellement, autonome.

Cette approche est de plus en plus utilisée pour modéliser des
systèmes complexes. L'avantage principal de ce type de modélisation
est sa simplicité de mise en place pour le modélisateur. Il suffit de
décrire le comportement de chaque agent en fonction de son
environnement. Ce sont souvent des rôles simples et issus de
l'observation (dans le cas des modélisations en biologie). Cependant
ce type d'approche autorise une certaine complexification aussi bien
du comportement des agents que des interactions entre agents et avec
leur environnement.

Ainsi en décrivant le nucléole ainsi que le comportement des molécules
qu'il contient, il est possible d'appliquer une approche multi-agents
(Annexe \ref{diagSMA}).

\section{Prototypage de l'interface}

\NQ a deux rôles principaux. Le premier est de permettre la
communication avec la base de données. Le second, ayant trait à la
simulation, est lui-même subdivisé en deux : la simulation en
elle-même ainsi que l'affichage des résultats de la simulation.

L'interface est donc divisée en fonction de ces trois fonctions
principales : 

\begin{itemize}
\item Le premier onglet (Fig. \ref{img_mock1}) est l'interface de communication avec la base de données : l'utilisateur peut y ajouter, modifier ou supprimer des données dans la base.
\item Le second onglet (Fig. \ref{img_mock2}) permet le paramétrage de
  la simulation et son démarrage.
\item Le troisième onglet (Fig. \ref{img_mock3}) ne s'active qu'en
  présence de résultats. Dans ce cas, il permet la génération des
  résultats sous différentes formes (graphique et tableau).
\end{itemize}

La visualisation de la simulation est une tâche lourde qui pourrait,
par la suite, demander à être désactivée à la demande. Lors du
démarrage de la simulation, la visualisation se fait donc dans une
fenêtre séparée de la fenêtre principale.

\image{mock1}{Premier onglet : interface avec la base de donnée.}{1}
\image{mock2}{Second onglet : paramétrage et lancement de la simulation.}{1}
\image{mock3}{Troisième onglet : génération et mise en forme des résultats.}{1}

%%% Local Variables: 
%%% mode: latex
%%% TeX-master: "../main"
%%% End: 
