\begin{center}\section*{Abstract}\end{center}

\paragraph{}
Les cellules eucaryotes ont la particularité d'être subdivisées en
structures spécialisées dont le noyau, lieu de stockage de
l'information génétique.  Cet organite est lui même organisé en trois
zones: l'euchromatine, l'hétérochromatine et le nucléole.  C'est ce
dernier qui est au coeur de ce projet.

Le nucléole, se trouve être le centre de biosynthèse des ribosomes.
Son apparence varie en fonction de son activité.  Lors d'une biogenèse
des ribosomes très active, le nucléole s'organise en trois composantes
majeures visibles en microscopie électronique : les centres
fibrillaires entourés par le composant fibrillaire dense lui même
enchâssé dans le composant granulaire. Le nucléole joue un rôle
capital dans l’organisation nucléaire. Ainsi, des mutations au niveau
de gènes codant pour des protéines nucléolaires ont été associées à
des maladies humaines.

\paragraph{}
Ce travail présente le développement d'une application (\NQ) conçu
dans le but d'accompagner les biologistes lors de l'etude du nucléole.

\paragraph{}
A ce jour, il n'existe encore aucune base de données relative au
nucléole. De ce fait, l'un des objectifs de \NQ est de proposer une
interface de saisie et de consultation des informations disponibles
sur le nucléole (images de microscopie, données qualitatives et
quantitatives sur ses molécules). Ces informations sont mises en
relation avec la partie simulation.

Le nucléole reste difficilement observable, sa modélisation peut donc
permettre de mieux comprendre son fonctionnement et d'apporter des
pistes pour les études pathologiques. C'est pourquoi, un autre des
objectifs de \NQ est la mise au point d'une modélisation de l'activité
du nucléole tout en proposant un suivi en temps réel de la simulation
via une visualisation en 3D. À l'issue de cette simulation, des
résultats sont proposés à l'utilisateur sous forme de graphiques ou de
tableaux.

%%% Local Variables: 
%%% mode: latex
%%% TeX-master: "../main"
%%% End: