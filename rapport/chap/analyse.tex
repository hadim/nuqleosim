\chapter{Analyse}

\section{Contexte}

\subsubsection{Généralités sur le nucléole}

Domaine nucléaire dynamique, le nucléole est le centre de synthèse des
ribosomes, à l'origine de la synthèse des protéines. Son état reflète
le niveau d'activité de la cellule. En effet, suivant la phase du
cycle cellulaire dans laquelle la cellule se trouve, le nucléole est
plus ou moins actif. Il peut même venir à disparaître totalement lors
de la mitose.

\subsubsection{Biogenèse des ribosomes}

Un ribosome comprend deux sous-unités, une petite et une grande. Ces
sous-unités sont des macro-molécules constituées d'ARN et de
protéines.

L'ADN présent dans le nucléole correspond aux gènes codant pour cet
ARN ribosomique ou ARNr (voir la figure \ref{img_biogenese.png}). Il y
est transcrit en pré-ARNr par l'ARN polymérase I. Le pré-ARNr est
ensuite scindé en 3 ARNr durant sa maturation.

Parallèlement, les gènes des protéines ribosomiques sont transcrits en
ARN messagers par l'ARN polymérase II dans le noyau. Cet ARNm est
alors exporté vers le cytoplasme où il sera traduit en protéines.  Les
protéines ribosomiques ainsi produites sont alors importées vers le
nucléole où elles seront associées aux ARNr pour former les
sous-unités ribosomiques.

Ces dernières sont enfin exportées vers le cytoplasme où elles
pourront remplir leur fonction.

\image{biogenese.png}{Biogenèse des ribosomes}{0.7}

\subsubsection{Morphologie du nucléole}
Le nucléole ne possède pas de membrane.  Cependant, chez les
eucaryotes supérieurs, il est généralement composé de trois régions
morphologiquement distinctes et visualisables en microscopie
électronique (Fig. \ref{img_microNuc}) :
\begin{itemize}
	\item un ou plusieurs centre(s) fibrillaire(s),
	\item un centre fibrillaire dense (ou composant fibrillaire dense) autour du premier centre fibrillaire, 
	\item et une région granulaire (ou composant granulaire) en périphérie. 
\end{itemize}
Selon les organismes, les proportions des volumes de ces trois compartiments peut varier.
Les séparations entre ces domaines ne sont pas nettes. 
Chez les plantes, des cavités (ou vacuoles) sont retrouvées en plus de ces territoires. 

\paragraph{Le centre fibrillaire :} il ne représente que 1 à 2 \% du volume total du nucléole. 
Il contient de l'ARN polymérase I et des facteurs de liaison amont (ou UBF pour Upstream Binding Factor, en anglais). 

\paragraph{Le centre fibrillaire dense :} il représente environ 17 \% de la fraction nucléolaire. 
Son volume reflète à peu près l'engagement nucléolaire dans la biogenèse des ribosomes. 

\paragraph{La région granulaire :} elle occupe environ 75 \% du volume total du nucléole. 
La granulosité de cette région périphérique est due à la présence de particules ribonucléoprotéiques. 

\image{microNuc}{Structure du nucléole.}{0.7}

\subsubsection{Maladies génétiques liées au nucléole}

Des mutations au niveau de gènes codant pour des protéines
nucléolaires sont associées à des maladies humaines. La variété des
maladies associées telles que le syndrome de Werner, le syndrome de
Treacher Collins ou encore la dyskératose congénitale, indique
l'importance fondamentale du rôle du nucléole. La compréhension de sa
dynamique ainsi que de sa structure est un premier pas vers le
développement de nouvelles thérapies.

\subsubsection{Les outils d'étude du nucléole}

Le biologiste qui étudiait le nucléole avait pour habitude de
répertorier l'ensemble des molécules étudiées dans un fichier de type
tableur (Excel ou OpenOffice Cacl). Par ailleurs, il n'existe pas
encore de simulation \cit{in silico} spécifique au nucléole.

\NQ a donc pour objectif d'assister le biologiste dans son travail, en
lui offrant une alternative à sa méthode de stockage de
l'information, ainsi qu'un outil lui permettant de réaliser des
simulations afin de mieux préparer les expériences en laboratoire.


\section{Besoins fonctionnels}

\subsubsection{Stockage de données liées au nucléole}
Les données qualitatives et quantitatives ainsi que les images de
microscopie électronique récupérées sur les objets biologiques du
nucléole doivent être conservées dans une base de données organisée et
accessible. Les objets biologiques concernés sont les principaux
constituants du nucléole soit des protéines, de l'ADN, des ARN, des
ribosomes et l'ensemble des réactions qui se déroulent dans cette
ultra-structure. Ces données devront être standardisées afin d'être
aisément exportables dans d'autres bases de données.

Il doit être possible de saisir, de visualiser et de modifier les
données contenues dans cette base. Les objets biologiques du nucléole
doivent être décrits de manière organisée afin de faciliter la
recherche et l'accès de ces objets. Un lien vers des bases de données
comme \href{http://expasy.org/sprot}{SwissProt} ou
\href{http://www.pdb.org/}{Protein Data Bank} par exemple pourra être
mis en place. Les images de microscopie électronique recueillies sur
le nucléole devront être regroupées et gérées par cette banque de
données.

Pour conserver les données accumulées et pouvoir les mettre à jour ou
simplement les consulter, la banque de données doit être sauvegardée à
chaque modification ou fermeture de l'application. Elle doit également
être chargée à chaque lancement de l'application.

\subsubsection{Simulation du nucléole}
Dans un second temps l'application doit gérer une simulation sur le
nucléole. Pour cela la saisie des paramètres doit être
intuitive. L'utilisateur doit également pouvoir utiliser commodément
les données contenues dans la banque de données décrite ci-avant. Une
fois la simulation initiée celle-ci doit être visualisable en temps
réel par l'utilisateur sans pour autant nuire à l'utilisation du reste
de l'application. L'utilisateur doit également gérer la simulation
lancée grâce à des fonctions de \cit{pause}, \cit{lecture} et
\cit{stop}. Une fois la simulation terminée les résultats s'affichent
pour permettre une analyse. Les résultats doivent pouvoir être traités
sous forme de tableaux ou de graphes personnalisables que
l'utilisateur peut régénérer à tout instant. Les résultats doivent
également pouvoir être exportés.

\section{Besoins non fonctionnels}
Ce projet devra également répondre à un certain nombre de besoins non
fonctionnels permettant une utilisation intuitive. En particulier
l'application doit être ergonomique et intuitive pour l'utilisateur.

\subsubsection{Portabilité}

Afin de faciliter la diffusion de \NQ, il devra pouvoir fonctionner
dans un maximum d'environnements d'exécution. Pour cela, il faudra
utiliser pour créer cette application un langage et des bibliothèques
disponibles pour GNU/Linux, Windows et MacOSX. Enfin, pour que la
simulation puisse être affichée en temps réel et en 3D dans tous ces
environnements, un outil de visualisation 3D multi-plateforme sera
nécessaire.

\subsubsection{Robustesse}

L'utilisateur de \NQ sera amené à saisir des données à deux
niveaux.  Le premier correspond à l'interface de saisie des
informations de la base de données : il faudra contrôler le format
des fichiers importés, ainsi que la nature des informations données
pour certains champs (liens, concentrations etc.).

Le second porte sur le paramétrage de la simulation et de son analyse.
Pour limiter les erreurs de saisie, la sélection de leurs paramètres
se fera à l'aide de listes déroulantes contenant des données présentes
dans la base.

Dans le cas oû l'application rencontrerait un problème, lors du
chargement d'un fichier ou lors de la simulation par exemple,
l'utilisateur devra en être averti par un message lui indiquant le
type d'erreur.

\subsubsection{Modularité et extensibilité}

L'application comprend deux parties : stockage de données et
simulation. Elle sera donc organisée en modules. Cela permettra de
répartir le travail de développement et éventuellement de réutiliser
ou ajouter des portions de code dans l'avenir.

\subsubsection{Performances}

Lorsqu'une simulation est en cours, elle ne doit pas bloquer
l'interface qui doit demeurer fluide. Par ailleurs, il pourrait être
intéressant d'activer l'utilisation du GPU pour la simulation si
celui-ci est détecté. Cette opération doit cependant rester
transparente pour l'utilisateur.

\section{Etat de l'art}

\subsection{Bases de données relatives au nucléole}

Les bases de données spécialisées sur le nucléole sont peu
répandues. Voici deux bases de données qui possèdent des informations sur
les molécules du nucléole :

\subparagraph{NOPdb: Nucleolar Online Proteomics Database \cite{Leung2006}: }
\url{http://www.lamondlab.com/NOPdb3.0/}. NOPdb est une base de données du
protéome humain du nucléole. Les données sont issues de la
spectrométrie de masse et la base contient à ce jour plus de 50 000
peptides correspondant à environ 4500 protéines humaines du
nucléole. L'API semble accessible sur demande.

\subparagraph{LOCATE \cite{Sprenger2007}: }
\url{http://locate.imb.uq.edu.au/}. LOCATE est une base de données
axée sur la localisation subcellulaires des protéines de la souris et
de l'Homme. Les localisations ont été déterminées par
immunofluorescence à haut débit. Les protéines du nucléole sont au
nombre de 1380 pour la souris et 1623 pour l'Homme. La base de données
complète regroupe environ 60 000 protéines.

\subsection{Les outils de modélisation}

Il n'existe pas à ce jour d'outil capable de simuler l'activité du
nucléole en prenant en compte toute sa dynamique et ses
caractéristiques. On pourra cependant citer une application telle que
\href{http://ccl.northwestern.edu/netlogo/}{NetLogo} \cite{Sklar2007}
qui est un environnement de modélisation orienté système multi-agent.

%%% Local Variables: 
%%% mode: latex
%%% TeX-master: "../main"
%%% End: 
