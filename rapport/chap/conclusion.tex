\chapter*{Conclusion}
\addcontentsline{toc}{chapter}{Conclusion}

Le nucléole, domaine nucléaire dynamique, est le centre de synthèse
des ribosomes. Son activité reflète l'équilibre entre le niveau de
synthèse des ARN ribosomiaux (ARNr), l'efficacité de la maturation des
ARNr et le transport des sous-unités ribosomiques vers le
cytoplasme. Le nucléole est un domaine très dynamique dont l’apparence
varie en fonction de son activité. En effet, quand la biogenèse des
ribosomes est active, le nucléole est organisé en trois composantes
majeures visibles en microscopie électronique : les centres
fibrillaires (FC) sont des régions claires dont la taille varie de 0,1
à 1 $\mu m$; ils sont entourés par le composant fibrillaire dense
(DFC), dont le nom est directement lié à son aspect contrasté et à sa
texture ; ces deux composants sont enchâssés dans le composant
granulaire (GC) constitué de granules de 15 à 20 nm. Le nucléole est
un domaine nucléaire multifonctionnel qui joue un rôle important dans
l’organisation nucléaire. Des mutations au niveau de gènes codant pour
des protéines nucléolaires sont associées à des maladies
humaines. C'est pourquoi il est actuellement envisagé de modéliser
l'activité nucléolaire pour pouvoir comprendre ses mécanismes et
apporter des pistes pour les études pathologiques.

Le projet présenté permet de modéliser dans différentes conditions les
diverses étapes de la biogenèse des ribosomes. \NQ est une application
comportant trois parties majeures. La première, la base de données,
permet le recensement de molécules, protéines ou ARN. Ainsi
l'utilisateur peut avoir accès aux différentes données générales sur
la dite molécule mais aussi aux données qu'il aura lui-même consignées
auparavant. Cette base de données peut donc être mise à jour par
l'utilisateur dès que ce dernier le désire. De plus, il est possible
d'accéder directement à l'ensemble des connaissances répertoriées sur
la fiche de cette molécule dans une base de données internet via son
identifiant dans la base de données en question. Dans la deuxième
partie de l'application, l'utilisateur peut simuler les mouvements des
molécules au sein d'un nucléole. Il peut pour cela définir le nucléole
lui-même ainsi que les molécules qui vont intervenir dans la
simulation. Une fois la simulation terminée les résultats peuvent être
affichées différemment sous forme de graphique ou de tableau grâce à
la troisième et dernière partie.

Néanmoins, de nouvelles améliorations peuvent être apportées au
programme. En effet, la modularité du code permet d'ajouter aisément
de nouveaux paramètres à la modélisation, mais aussi en parallèle dans
l'interface, comme la spécification par l'utilisateur du rôle d'une
molécule dans les étapes de maturations des ARNr. Il pourrait être
également envisagé, à partir des résultats, de créer automatiquement
une \og{} fiche \fg{} de l'expérience simulée pour la conserver dans
la base de données de \NQ. Enfin, il pourrait être intéressant
d'étendre le module de modélisation afin d'effectuer les calculs sur
le GPU et ainsi gagner en performance.

%%% Local Variables: 
%%% mode: latex
%%% TeX-master: "../main"
%%% End: 
