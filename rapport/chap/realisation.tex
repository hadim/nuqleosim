\chapter{Réalisation}

\section{Langage et bibliothèques retenus}

La nécessité d'intégrer une simulation et une visualisation 3D en
temps réel a conduit a développé \NQ en C++. Cependant, la complexité
de développement qu'implique parfois le C++ est largement compensée
par l'utilisation du framework Qt. En effet, dans ce projet Qt es
utilisé en tant que framework et non en tant qu'une simple
bibliothèque GUI comme GTK ou wxWidget. L'architecture de \NQ repose
donc princiapelemnt sur Qt, cependant certain module comme celui de la
modélisation reste indépendant de Qt afin de le rendre facilement
portable en supprimant la dépendance à Qt.

Qt implique des temps de compilation relativement long et des
executables souvent de grosses taille. Cependant, ceci est largement
compensé par plusieurs avantages tel que la rapidité d'execution des
applications Qt, la rapidité de développement grâce à une API très
bien documenté et la présence d'une large communauté. De plus, Qt est
totalement multiplateforme.

Voici une liste des technologies utilisées lors de la réalisation de \NQ.

\paragraph{Framework Qt}(\url{http://qt.nokia.com/products/}) : est
un framework orienté objet et développé en C++ par Qt Development
Frameworks, filiale de Nokia. Elle offre des composants d'interface
graphique (widgets), d'accès aux données, de connexions réseaux, de
gestion des files d'exécution, d'analyse XML, etc. En plus d'être très
complet comme framework, un autre avantage est son aspect multi
plateforme. En effet, par simple recompilation du code source, un
programme Qt pourra être porté vers des OS tel que GNU/Linux, Windows
et Mac OS X.

\paragraph{libQGLViewer}(\url{http://www.libqglviewer.com/}) : est une
bibliothèque basé sur Qt qui facilite la création de scène OpenGL.  Il
offre certaines fonctionnalités typiques des visonneurs 3D, comme la
possibilité de déplacer la caméra en utilisant la souris, ce qui
manque à la plupart des autres API. Les autres caractéristiques
comprennent la manipulation de la souris, la superposition de widget
vectorisé aux scènes 3D, la sélection d'objets. Il permet de créer des
applications 3D complexes rapidement. Cette bibliothèque est utilisé
pour la visualisation en temps réel lors de la simulation.

\paragraph{Qwt}(\url{http://qwt.sourceforge.net/}) : est une
bibliothèque basé sur Qt qui fournit un ensemble de widget destiné à
faciliter l'affichage de graphique destiné aux applications techniques
et scientifiques. Elle est utilisé pour l'affichage des résultats
issus de la modélisation.

\paragraph{XML}(\url{http://www.w3.org/XML/}) : est un langage
informatique de balisage générique. Il sert essentiellement à
stocker/transférer des données de type texte Unicode, structurées en
champs arborescents. Il sera utilisé pour le stockage de la base de
données d'objets biologiques. En effet, cette solution contrairement à
des SGBD plus classiques tel que MySQL a l'avantage d'être indépendant
(pas besoin de serveur SQL) et de faciliter l'échange de base de
données (simple fichier XML). La cohérence de la base de données sera
assurée par un fichier XML Schema qui est un langage de description de
format de document XML permettant de définir la structure d'un
document XML.

\paragraph{CMake}(\url{http://www.cmake.org/}) : est est un \og moteur
de production \fg{} multiplate-forme. Il permettra de recompiler
facilement l'ensemble du code source sous différents OS. Il possède
aussi des outils de génération de fichiers d'installation (.deb, .rpm,
.exe, .tar.gz).

\paragraph{Git}(\url{http://git-scm.com/}) : est est un logiciel de
gestion de versions décentralisée. Son choix a été motivé par le fait
qu'il soit open-source d'une part et d'une autre part pour sa grande
flexibilité. Git est capable de gérer des cas complexes de conflits de
fusion entre plusieurs versions. De plus, Git, créé par Linus Torvald,
est aujourd'hui utilisé pour la gestion des sources du kernel de
Linux. On imagine donc que si cet outil convient à la gestion d'un
aussi gros projet (11,5 millions de lignes de codes) avec autant de
collaborateurs, il conviendra parfaitement à un projet plus modeste.

\section{Diagramme de classe}

L'architecture est composé de plusieurs modules qui communique entre
eux principalement via le mécanisme de signal et slot de Qt (un tel
mécanisme existe aussi dans d'autre bibliothèque connu tel que Boost).

Une version simplifié et une version détaillée sont présentes en
annexe de ce rapport (\ref{diag_class} et
\ref{diag_class_detail}). Voici un bref description des principaux
modules :

\begin{itemize}
\item Le modèle de donnée est un ensemble de classe destiné à
  modéliser les données manipulées lors de l'interfaçage avec la base
  de donnée.
\item \texttt{BioParser} est une classe abstraite qui permet de
  facilement écrire des classes capables de parser n'importe quel
  format de fichier biologique. Pour le moment, \NQ est capable de
  parser les fichiers de type PDB.
\item Le module \texttt{DataUI} et ses classes associés sont un
  ensemble de classe widget chargé de l'interfaçage avec la base de
  donnée.
\item Le module de base de donnée permet la communication du reste de
  l'application avec n'importe quel type de base de donnée via une
  classe abstraite \texttt{DBAccess}. A ce jour, \NQ repose sur une
  base de donnée de type XML.
\item Le module \texttt{SimuUI} et ses classes associés sont un
  ensemble de widget chargé du paramétrage de la simulation. Son rôle
  est aussi de gérer le lancement et le suivi de la modélisation ainsi
  que la mise à jour de la visualisation.
\item \texttt{Computing Module} est chargé de la simulation du
  nucléole. Il ne dépend pas de Qt afin de le rendre facilement
  portable. Le système d'héritage permet, dans l'avenir, l'ajout de
  manière de modéliser supplémentaires. Pour le moment, les calculs
  sont effectués sur le CPU, mais il a été prévu de permettre l'ajout
  de classe effectuant les calculs sur le GPU via différentes
  librairies tel que OpenCL ou encore CUDA.
\item \texttt{3D Module} est chargé d'afficher l'espace cellulaire
  représentant le nucléole lors de la simulation.
\item \texttt{AnaUI} est un ensemble de widget capable de générer des
  résultats provenant d'une modélisation précédente.
\end{itemize}

\section{Base de données}

Dans un premier temps \NQ doit permettre d'enregistrer des données
relatives à des molécules ainsi qu'à des expériences. Une base de
données à donc été mise en place. Elle permet dans un premier temps la
saisie par l'utilisateur des différents champs d'intérêt, que ce soit
pour une molécule ou une expérience (Fig. \ref{img_diagDB}. Dans le
cas de la partie de la base de données gérant les molécules,
l'utilisateur peut spécifier :

\begin{itemize}
	\item \texttt{id} : l'identifiant dans la base de données de \NQ
	\item \texttt{name} : le nom de la molécule
	\item \texttt{type} : le type moléculaire : ADN, ARN, protéine, ribosome, molécule
	\item \texttt{sequence} : la séquence de la protéine (quelque soit son format)
	\item \texttt{length} : la longueur de la séquence
	\item \texttt{comments} : les éventuels commentaires de l'utilisateur
	\item \texttt{concentration} : la concentration de la molécule dans le nucéole
	\item \texttt{embl\_id} : l'identifiant de la molécule dans l'EMBL avec un lien vers la fiche
	\item \texttt{uniprot\_id} : l'identifiant de la molécule dans UNIPROT avec un lien vers la fiche
	\item \texttt{pdb\_id} : l'identifiant de la molécule dans la PDB avec un lien vers la fiche
\end{itemize}

Ainsi, il est possible de faire des modifications à tout instant. De
plus, des liens vers la page web de la molécule dans diverses bases de
données biologiques ont été mis en place pour compléter les
caractéristiques des molécules. Afin de faciliter l'import de
molécule, il a été mis en place un système de téléchargement
automatique de fichier PDB via l'id PDB.

Concernant les expériences, l'implémentation est similaire, si ce
n'est que les noms des champs à compléter est différent :

\begin{itemize}
	\item \texttt{id} : l'identifiant dans la base de données de \NQ
	\item \texttt{name} : le nom de la molécule
	\item \texttt{type} : le type moléculaire : ADN, ARN, protéine, ribosome, molécule
	\item \texttt{comments} : les éventuels commentaires de l'utilisateur
	\item \texttt{date} : l'identifiant de la molécule dans UNIPROT avec un lien vers la fiche
	\item \texttt{author} : l'identifiant de la molécule dans la PDB avec un lien vers la fiche
\end{itemize}

Cette base de données doit être rechargée à chaque ouverture de \NQ et
enregistrée à chaque modification. Voici, un aperçu d'un fichier XML
de la base de donnée de \NQ vide.

\begin{lstlisting}
<nucleolus_db>
	<experiments>
		<experiment>
			<author/>
			<comments/>
			<date/>
			<name/>
			<images>
				<image>
					<comments/>
					<id/>
					<name/>
					<path/>
				<image>
			<images>
	</experiments>
	<molecules>
		<molecule>
			<coments/>
			<default_concentration/>
			<name/>
			<sequence/>
			<sequence_length/>
			<type/>
			<embl_id/>
			<pdb_id/>
			<uniprotkb_id/>
		<molecule>
	<molecules>
</nucleolus_db>
\end{lstlisting}

Enfin, il est possible de compléter les \texttt{experiences} par une
ou des images. En effet, tout comme la saisie et la consultation des
informations relative à une expérience ou une molécule, l'ajout
d'images est pris en charge par la base de données.

\section{Modélisation du nucléole}

Comme vu lors de la conception, la modélisation du nucléole est basé
sur une approche multi-agents. Les différentes molécules ou systèmes
qui entrent en jeu peuvent être assimilés à des agents qui ont des
caractéristiques, qui réalisent des actions et qui interagissent entre
eux ou avec leur environnement.

\subsection{Environnement}

Afin de modéliser le nucléole où se déroule la simulation, un
environnement tridimensionnel a été mis en place. Le nucléole est
modélisé comme étant un parallélépipède rectangle dont les trois
dimensions sont renseignées par l'utilisateur. En outre, les trois
composantes du nucléole que constituent le FC, le DFC et le GC sont
elles aussi assimilées à des structures parallélépipédiques
tridimensionnelles dont les dimensions sont calculées via un
pourcentage de l'environnement nucléolaire total. On retrouve ainsi au
centre le FC entouré du DFC lui même entouré du GC
(Fig. \ref{img_schemaSimu}. Même si aucune membrane physique ne
délimite ces différentes composantes dans la \og{} réalité \fg{}, afin
de mieux les distinguer au cours de la simulation, il est possible de
faire apparaître chaque compartiment dans une couleur distincte des
deux autres.

\subsection{Agents}

Une fois l'environnement mis en place, il faut placer les différents
éléments qui vont entrer en jeux dans la simulation. De plus, chaque
élément possède des contraintes de par son type même ou imposées par
l'utilisateur. Ainsi, les différents agents sont placés aléatoirement
tout en respectant certaines règles :

\begin{itemize}
	\item deux éléments ne peuvent pas être en même temps sur une même
  \og{} case \fg{}.
	\item un élément ne peut pas être placé dans une composante dans
      laquelle il ne prend pas part.
\end{itemize}

Un nombre déterminé de chaque agent doit être placé pour initialiser
la simulation. Pour cela l'utilisateur précise pour chaque molécule
les composantes dans lesquelles il veut les situer et dans quelles
proportions. Ainsi, une molécule peut se déplacer dans les trois
compartiments, mais n'être présente au début de la simulation que dans
l'un des trois par exemple (Annexe \ref{diagSMA}). Cependant si une
molécule ne participe pas à la constitution d'une composante, il ne
peut pas y avoir de molécules de ce type dans cette composante.

Au cours de la simulation, les agents peuvent se déplacer au sein de
l'environnement. Dans le but de ne favoriser aucun agent devant un
autre, ils sont déplacés chacun leur tour dans un ordre aléatoire. De
plus, chaque molécule possède sa propre probabilité de
déplacement. Ainsi, une fois un agent sélectionné aléatoirement, il
est déplacé dans une case voisine, respectant les contraintes
appliquées à l'agent, aléatoirement.

\image{schemaSimu}{Interactions des différents agents entre eux et
  avec leur environnement.}{1}

Lors des déplacements on distingue les molécules protéiques des
molécules d'ARN. En effet, les ARN subissent certaines étapes de
maturation avant de sortir du nucléole et du noyau. Pour modéliser ce
comportement une probabilité de maturation d'un ARN à proximité d'une
protéine a été définie. L'ARN maturé se distingue de sa forme précoce
de par sa capacité à accéder à la composante nucléolaire suivante ou
sa sortie du nucléole pour aller constituer les ribosomes.

\subsection{Déroulement}

Lorsque une simulation est initiée, les données sont enregistrées à
chaque pas de temps. Soit : pour chaque type de molécules, leur
quantité dans chaque composante est conservée. En parallèle, leurs
mouvements sont calculés pour chaque pas de temps et transmis au
module gérant la visualisation pour permettre l'affichage en trois
dimensions de la simulation en cours (Fig. \ref{img_debutSimu}). Cet
échange mis en place permet également de mettre en pause une
simulation en cours ou de la stopper. Une fois le temps de simulation,
spécifié par l'utilisateur, terminé l'ensemble des données rassemblées
peuvent être traitées.

\subsection{Résultats}

Dans un dernier temps, une fois la simulation terminée, l'utilisateur
peut sélectionner parmi les diverses molécules présentes dans la
simulation, celles dont il veut observer l'évolution. Il peut alors
décider d'afficher les résultats sous forme de graphe ou bien dans un
tableur qu'il pourra exporter respectivement au format PNG ou CSV
(Fig. \ref{img_grapheResu} et Fig. \ref{img_tabResu}).

\section{Fonctionnement de l'interface}

\image{accueil}{Interface d'ouverture de \NQ.}{1}

Tout comme l'implémentation du projet peut être scindée en trois
parties principales, l'interface permettant à l'utilisateur d'utiliser
l'application comprend à son ouverture trois parties principales,
chacune associée à un onglet (Fig. \ref{img_accueil}).

\subsection{Onglet "Database"}

\image{BDMolecule}{Interface de saisie ou de modification de données
  sur une molécule.}{1}

\image{BDExp}{Interface de saisie ou de modification des données d'une
  expérience.}{1}

Le premier onglet de \NQ permet d'accéder à la base de données
constituée par les utilisateurs. L'interface permet la distinction
entre les molécules pouvant prendre part à une simulation ou une
expérience (Fig. \ref{img_BDMolecule}), des expériences elles-mêmes
(Figure ref BQExp). Pour chaque molécule, il est possible de rentrer
son identifiant, son nom, sa taille, sa concentration, son identifiant
dans diverses bases de données ainsi que sa séquence et de préciser sa
nature dans une liste de molécules pré-rentrée (ADN, ARN, protéique,
etc). Un lien est alors mis en place pour permettre d'accéder
directement aux informations sur les bases de données internet
spécifiées. Concernant l'enregistrement d'une expérience, il est
possible, de même, de préciser divers champs comme l'identifiant de
l'expérience, l'expérimentateur et la date. Dans le cas des
expériences, des images peuvent être associées. Enfin, pour les deux
un champs \og{} \texttt{comments} \fg{} a été intégré pour permettre à
l'utilisateur de rentrer des commentaires ou ajouter des données dont
les champs n'existent pas encore.

\subsection{Onglet "Simulation"}
\image{paramSimu}{Interface de paramétrage d'une simulation.}{1}

Le deuxième onglet de \NQ permet le paramétrage d'une simulation
(Fig. \ref{img_paramSimu}). Ce paramétrage peut être scindé en deux en
distinguant les paramètres se rapportant au nucléole, des données se
rattachant aux molécules intervenant dans la simulation. Une
séparation spatiale sur l'interface est alors opérée. En haut les
grandeurs spécifiant le nucléole modélisé et en dessous les variables
discriminant les molécules. Les données relatives aux molécules sont
accessibles en sélectionnant une molécule parmi celles de la base de
données de \NQ. L'utilisateur doit alors sélectionner (en cochant la
case dédiée) celles qui l'intéressent, puis spécifier leurs
différentes concentrations dans chaque composante ainsi que les
composantes auxquelles elles peuvent avoir accès. Toute molécule
sélectionnée est alors surlignée en rouge dans la liste, pour
permettre à l'utilisateur d'avoir une vue d'ensemble de sa simulation.

\image{debutSimu}{Fenêtre de suivi en temps réel de la simulation.}{1}

Une fois la simulation paramétrée, l'utilisateur peut l'initier en
cliquant sur le bouton \og{} \texttt{Launch} \fg{}. Une nouvelle
fenêtre s'ouvre alors (Fig. \ref{img_debutSimu}). La simulation en
cours peut alors être mise en pause ou stoppée à tout instant. Afin de
différencier aisément les différentes molécules en jeu, une couleur
différente leur est attribuée dans la modélisation (une légende
permettant de faire le lien). De plus dans l'optique de distinguer les
ARN des autres molécules et pour rester plus ou moins en accord avec
la réalité, les sphères modélisant les ARN sont plus petites. Enfin
pour laisser une certaine marge de liberté a l'utilisateur, ce dernier
peut moduler certains aspects de la simulation. Effectivement, il peut
modifier la taille des molécules et la vitesse de la simulation. Il
peut faire apparaître indépendamment les différentes composantes (Fig.
\ref{img_duringSimu}) ou encore ne visualiser que les molécules
présentes dans une ou plusieurs composantes. Pour terminer, des
fonctions de zoom, de contrôle de la transparence des composantes,
d'affichage des axes ou encore d'une grille sont disponibles.

\image{duringSimu}{Visualisation possible des différentes composantes.}{1}

\subsection{Onglet "Results"}
\image{paramResu}{Onglet de paramétrage de la visualisation des
  résultats de la simulation.}{1}

Lorsque la simulation est terminée le troisième et dernier onglet
\texttt{résultat} est alors opérationnel. C'est-à-dire que
l'utilisateur peut organiser les résultats issus de la simulation
comme il le souhaite dans un ou divers graphes
(Fig. \ref{img_grapheResu}) ou tableaux (Fig. \ref{img_tabResu}). Afin
de rendre l'interface la plus intuitive possible, la sélection des ARN
à analyser se fait par compartiments. Chaque graphe ou tableau est
alors généré dans un nouvel onglet, dans la partie basse de l'onglet,
et peut être exporté aux formats PNG ou CSV par un simple clic sur le
bouton approprié.

\image{tabResu}{Exemple de tableau de résultats.}{1}

\image{grapheResu}{Exemple de graphe de résultats.}{1}

%%% Local Variables: 
%%% mode: latex
%%% TeX-master: "../main"
%%% End:
