\chapter*{Introduction}
\addcontentsline{toc}{chapter}{Introduction}

La cellule est l'unité structurale constitutive des êtres
vivants. L'existence ou non d'un noyau permet la distinction de deux
types cellulaires : respectivement eucaryote et procaryote. La cellule
eucaryote est subdivisée en structures spécialisées, appelées
organites, auxquelles appartient le noyau. Sa fonction principale est
le stockage de l'information génétique: l'ADN. Il est organisé en
trois zones: l'euchromatine, l'hétérochromatine et le nucléole. Ce
dernier, encore mal connu semble être impliqué dans de nombreuses
maladies. C'est pourquoi il est aujourd'hui de plus en plus étudié.

Afin de mieux comprendre les processus internes au nucléole, il existe
deux types d'approches. L'expérience en laboratoire malgré sa
pertinence biologique coûte encore cher et prends du temps. Son
alternative, la simulation \textit{in silico}, réduit fortement les
coûts à l'instar d'une pertinence biologique moindre. C'est pourquoi,
les outils de modélisation du nucléole sont à développer dans le
futur.

L'objectif de ce projet est de réaliser une application (\NQ)
regroupant la gestion d'une base de données relatives au nucléole
ainsi qu'un outil permettant la simulation de l'activité du
nucléole. Elle propose également un module d'analyse des résultats de
la simulation.

Dans un premier temps, le contexte biologique permettra de préciser
les attentes auxquelles répondra l'application en décrivant les
besoins fonctionnels et non fonctionnels. Ensuite, la conception et
l'implémentation qui en découle seront détaillées. Enfin, un cas
d'utilisation de \NQ sera présenté.

%%% Local Variables: 
%%% mode: latex
%%% TeX-master: "../main"
%%% End: 
